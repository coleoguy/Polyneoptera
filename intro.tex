\section{Introduction}
Chromosome number is one of the fundamental characteristics of a genome.
It is also the first information collected about most genomes. 
In fact, the first chromosome counts were recorded prior to the development of the chromosome theory of inheritance \citep{flemming1882}.
Despite this early start consistent rules governing the evolution of chromosome number across large clades remain elusive. 

Chromosome number can have broad impacts on gene transcription, recombination rates, and sex chromosome evolution. 
For instance, the presence of a third copy of chromosome 21 in humans can lead to approximately a four percent increase in gene expression across all chromosomes \citep{lockstone2007}.
Furthermore, in mouse, it has been shown that in chromosomes which have three copies, there is an average of 1.5-fold increase in expression of genes on that chromosome \citep{williams2008aneuploidy}. 
It has long been recognized that chromosome number should correlate with genome wide recombination rates \citep{stebbins1958}.
Crossing over in meiosis is required for proper segregation of chromosomes into gametes and ultimately determines the frequency of recombination.
The lower limit of the number of crossing over events is controlled by the number of chromosome arms in most species and by the number of chromosomes in some species \citep{dumont2017req}.
These requirements suggest that there should be a positive correlation between recombination rates and the number of chromosomes. 
This relationship between chromosome number and recombination has been suggested as a source of indirect selection on chromosome number in Hymenoptera (\citealt{sherman1979}; but see \citealt{ross2015}).
In fact, reduced recombination between loci (due to a fusion between an autosome and the sex chromosomes --- reducing chromosome number) has been implicated in the speciation of the Japan Sea stickleback \citep{kitano2012}. 
Finally, changes in chromosome number can have impacts on the evolution and behavior of sex chromosomes. 
All else being equal average chromosome size should be negatively correlated with the number chromosomes.
This can have important impacts on the fate of sex chromosomes.
A comparative study of Coleoptera has shown that species are more likely to lose the Y chromosome and transition from XY to XO if they have many small chromosomes rather than few larger chromosomes \citep{blackmon2015bioessay}.
It is thought this pattern is driven by the size of the region available for recombination between the X and Y chromosome during male meiosis and the frequency that males will produce aneuploid gametes\citep{blackmon2014}.
However, in some species this may be averted by cell cycle checkpoints that lead to apopstosis of cells that fail to properly segregate the sex chromosomes \citep{dumont2017par}.

In sexual species, it is often assumed that changes in chromosome number are underdominant \citep{white1973} -- heterozygotes have a reduced fitness. 
Perhaps the mostly widely known example of this is with hybrids between horses and donkeys where the offspring carries 32 chromosomes from the mother horse and 31 chromosomes from the father donkey. 
When this mule attempts to produce gametes the rearrangements that have occurred lead to gametes lacking a full set of all genes in the genome and thus they are sterile \citep{wodsedalek1916}. 
It should be noted this is not always the case.  %HB: find example from rodents get from Max King 1995 write 1-2 sentence here

Species of \textit{Mus} have been studied extensively for the impact on fitness with regards to chromosomal translocations.

It has been shown that in wild mice which are heterozygouse for a single fusion between chromosomes 16 and 17 [Rb(16.17)], there is no significant reduction in fertility and thus no reduction in fitness \citep{britton1990robertsonian}.

In addition, a study on hybrid sterility on Lemurs, some crosses between species with different chromosome numbers produced hybrids which had same level of spermatogenesis as the parental species and had no effect in their reproductive capabilities \citep{king1995}.   

However, we can imagine that changes in chromosome number might be less deleterious in asexual species since they do not have to pair with any other genome in the population.
Consistent with this asexual species have considerable variation in chromosome number. 
For example, \textit{Catapionus gracilicornis} in the family Curculionidae (weevils), has polyploid races with chromosome numbers of 22, 33, 44 and 55 \citep{lachowska1998}. 

To better understand the dynamics of chromosome evolution we have chosen to work with the insect clade Polyneoptera.
This group contains the orders Blattodea (including Isoptera), Mantodea, Dermaptera, Orthoptera, Phasmatodea, Plecoptera, Embiidina, and Notoptera.
These orders shows striking variation in chromosome number, sex determining systems and, and includes sexual and asexual species. 
We have assembled a large trait data set that includes chromosome number, sex chromosome system (SCS) , and reproductive mode.
We analyze this data in both a taxonomic and phylogenetic framework to determine the impact of sexual system on rates of chromosome number evolution, the source of transitions in SCSs, and identify differences in patterns of chromosome number evolution across orders.
 
