\section{Introduction}

Chromosome number is one of the fundamental characteristics of a genome.
It is also the first information collected about most genomes. 
In fact, the first chromosome counts were recorded prior to the development of the chromosome theory of inheritance \citep{flemming1882}.
Despite this early start, consistent rules governing the evolution of chromosome number across large clades remain elusive. 

Changes in chromosome number can happen due to several mechanisms.
We use the term fusion and fission to describe respectively a decrease or an increase of one in chromosome number.
However, these terms are simplifications and may represent multiple processes at the molecular level.
Reduction in chromosome number can happen through chromosomal fusion near the centromere region followed by loss of non-essential DNA, a process termed as Robertsonian translocation \citep{garagna1995}. 
% I've never heard of a "centromeric fusion" and I don't find anything like this in the paper you are citing. To my knowledge what we call fusions are robertsonian translocations the paper you have cited even points this out. Please read about them and ask me if any part of it doesn't make sense or point me to something specific that is making you say this. If not please rewrite this sentence to explain that what we call fusions are often robertsonian translocations with loss of some nonessential DNA
Redcutions can also happen through the fusion of two chromosomes at the telomeres followed by loss of one of the centromeres \citep{gordon2011mechanisms, miga2016}. %There has long been evidence for this as the orgin of human chromosome 2 but is thought to be kind of rare and I've never been too sure how much to consider it as important especially based on the early 90s data we had but Miga's paper looks at this with much more recent data in humans
In contrast, increases in chromosome number can occur due to simple chromosome fission in the centromere region \citep{moretti1984}.
Increases in chromosome number can also happen due to the duplication of an entire chromosome.
Changes in chromosome number of more than one can also occur.
Although rare in most animal groups, demiploidy describes an increase chromosome number by one-half. 
Demiploidy events can occur by the joining of haploid gamete with an unreduced diploid gamete \citep{hornsey1973}.
Finally, whole genome duplication can lead to a doubling of chromosome number \citep{beccak1970}.

These changes in chromosome number can have broad impacts on gene transcription, recombination rates, and sex chromosome evolution.
Presence of an extra copy of a chromosome can lead to both increases and decrease in gene transcription.  
For instance, the presence of a third copy of a chromosome can lead to approximately a four per cent increase in gene expression across all chromosomes and a 1.5-fold increase in expression of genes on that chromosome \citep{lockstone2007, williams2008aneuploidy}.
In \textit{Drosophila} females with three X chromosomes, expression of genes in each X chromosome is reduced such that the total X chromosome gene expression matches with that of females with two X chromosomes \citep{sun2013dosage}.
However, the presence of an extra copy of the X chromosome in \textit{Drosophila} females leads to a reduction in the expression of genes in the autosomes by 33\% \citep{sun2013dosage}. 

It has long been recognized that chromosome number should positively correlate with genome-wide recombination rates \citep{stebbins1958}.
furthermore, the frequency of recombination events and the proper segregation of chromosomes into gametes is determined by crossing over in meiosis.
The lower limit of the number of crossing over events is controlled by the number of chromosome arms in most species and by the number of chromosomes in some species \citep{dumont2017req}.
This relationship between chromosome number and recombination has been suggested as a source of indirect selection on chromosome number in Hymenoptera (\citealt{sherman1979}; but see \citealt{ross2015}).
In fact, reduced recombination between loci (due to a fusion between an autosome and the sex chromosomes --- reducing chromosome number) has been implicated in the speciation of the Japan Sea stickleback \citep{kitano2012}. 

Changes in chromosome number can have impacts on the evolution and behaviour of sex chromosomes. 
For instance, if chromosomes are broken into smaller chromosomes while keeping all else equal (e.g. genome size), the average chromosome size should be negatively correlated with the number of chromosomes.
This can have important impacts on the fate of sex chromosomes.
Comparative study of Coleoptera has shown that species are more likely to lose the Y chromosome and transition from XY to XO if they have many small chromosomes rather than a few larger chromosomes \citep{blackmon2015bioessay}.
It has been hypothesized that this pattern is driven by the size of the region available for recombination between the X and Y chromosome during male meiosis and the frequency that males will produce aneuploid gametes \citep{blackmon2014}.
However, in some species, this may be averted by cell cycle checkpoints that lead to apoptosis of cells that fail to properly segregate the sex chromosomes \citep{dumont2017par}.

In sexual species, it is often assumed that changes in chromosome number are underdominant \citep{white1973} -- heterozygotes have reduced fitness. 
Heterozygosity at the chromosome level can occur, for instance, when an organism receives two chromosomes from each parent, if one copy is a single fused chromosome and the other is two regular chromosomes or if one copy has a fissioned chromosome but not in the other.
Perhaps the most widely known example of this is hybridization between horses and donkeys where the offspring carries 32 chromosomes from the mother and 31 chromosomes from the father. 
When this mule attempts to produce gametes the rearrangements that have occurred lead to gametes lacking a full set of all genes in the genome and thus, they are sterile \citep{wodsedalek1916}. 
It should be noted that this fitness reduction is not always observed.
In wild mice which are heterozygous for a single fusion between chromosomes 16 and 17, there is no significant reduction in fertility and thus no reduction in fitness \citep{britton1990robertsonian}.
A large number of crosses in lemurs exhibit a full range of fitness effects of changes in chromosome number.
Four of twelve hybrids exhibit normal spermatogenesis while six of twelve exhibits reduced spermatogenesis while the final two hybrids exhibited major perturbations to spermatogenesis \citep{ratomponirina1988}.   
In contrast, one can hypothesize that changes in chromosome number might be less deleterious in asexual species since they do not have to pair with any other genome in the population.
Consistent with this asexual species have considerable variation in chromosome number.
In parthenogenetic beetle species, we find polyploid races within these species where the total chromosome number of these polyploid races is a multiple of the haploid number.
For example, in weevils (Coleoptera: Curculionidae), the species \textit{Catapionus gracilicornis} has polyploid races with chromosome numbers of 22, 33, 44 and 55 \citep{lachowska1998}. 

To better understand the dynamics of chromosome evolution we have chosen to work with the insect clade Polyneoptera which includes commonly known species such as termites, grasshoppers, and cockroaches.
This group contains the orders Blattodea (including Isoptera), Dermaptera, Embiidina, Mantodea, Notoptera, Orthoptera, Phasmatodea, and Plecoptera.
These orders show striking variation in chromosome number, sex chromosome systems (i.e. XX/XO, XX/XY or complex XY) and, includes sexual and asexual species. 
We have assembled a large trait dataset that includes chromosome number, Sex Chromosome System (SCS), and reproductive mode.
We analyze this data in both a taxonomic and a phylogenetic framework to determine the impact of the sexual system on rates of chromosome number evolution, the source of transitions in SCSs, and identify differences in patterns of chromosome number evolution across orders.