\section{Introduction}
Chromosome number is one of the fundamental characteristics of a genome.
It is also the first information collected about most genomes. 
In fact, the first chromosome counts were recorded prior to the development of the chromosome theory of inheritance \citep{flemming1882}.
Despite this early start consistent rules governing the evolution of chromosome number across large clades remain elusive. 

Chromosome number can have broad impacts on gene transcription, linkage, and meiosis. 
For instance, the presence of a third copy of chromosome 21 in humans can lead to approximately a four percent increase in gene expression across all chromosomes \citep{lockstone2007}. Furthermore, in mouse, it has been shown that in chromosomes witch have three copies, show an increase in gene expression by 1.5-folds on average \citep{williams2008aneuploidy}. 

Likewise, it has long been recognized that chromosome number should correlate with genome wide recombination rates \citep{stebbins1958}.
This expectation is driven by the nearly ubiquitous requirement for at least one crossover per chromosome per meiosis\citep{dumont2017req}.
This correlation has been suggested as a source of indirect selection on chromosome number in Hymenoptera \citep{sherman1979,ross2015}.
In fact, reduced recombination between loci has even been show to play a role in speciation in the Japan Sea stickleback \citep{kitano2012}. 

Finally, changes in chromosome number can have important impacts on meiosis. 
Crossing over in meiosis determines the proper segregation of chromosomes into gametes.
The lower limit of the number of crossing over events is controlled by the number of chromosome arms in most species and by the number of chromosomes in some species \citep{dumont2017req}.
Failure in at least one crossing over can form aneuploid gametes which in turn produces infertile offspring or can lead to death of the resulting zygote \citep{hassold2001err}.
However, few studies have investigated the role that sexual system may play in the evolution of chromosomes number.  
We hypothesize that the impact of changes in chromosome number are moderated by the sexual system of an organism. 
In sexual species, it is assumed that changes in chromosome number are underdominant \citep{white1973} - heterozygotes for changes in chromosome number have a reduced fitness. 
Perhaps the mostly widely known example of this is with hybrids between horses and donkeys where the offspring carries 32 chromosomes from the mother horse and 31 chromosomes from the father donkey. 
When this mule attempts to produce gametes the rearrangements that have occurred lead to gametes lacking a full set of all genes in the genome and thus they are sterile \citep{wodsedalek1916}. 

However, it should be noted this is not always the case. For instance, in the \textit{Chilocorus stigma} (Twice-stabbed ladybug) populations, irregular chromosome numbers are seen in both males and females \citep{smith1962}.  
In contrast, we can imagine that underdominance would not be an important concern in an asexual species. 
This is Because there is no involvement in fertilization of gametes from two parents in asexual reproduction.
Consistent with this asexual species have considerable variation in chromosome number. 
For example, \textit{Catapionus gracilicornis} in the family Curculionidae (weevils), consists polyploid races with chromosomes numbers 22, 33, 44 and 55 \citep{lachowska1998}. 
Therefor we predict that asexual species should show increased rates of chromosome number evolution when compared to sexual species.

The insect group Polyneoptera shows a significant variation in chromosome numbers, sex determining systems and, and includes species in both asexual and sexual reproductive types. 
Chromosome number in this order is as low as 8 in the genera \textit{Hemimerus} and \textit{Dichroplus} and as high as 98 in the species \textit{Mastotermes darwiniensis}.
Of the 359 genera in our dataset we find that 64 are variable for chromosome number with the genera \textit{Sipyloidea} in the order Phasmatodea being particularly variable having a range of 22 to 80. 
Furthermore, Polyneoptera includes XX/XO and XX/XY sex determination systems \citep{blackmon2016}. 
There is no a proper documentation on the impact of sex system on chromosome number changes in this group despite abundant data and variation. 

To explore dynamics of chromosome evolution we analyzed 783 karyotypes from the clade Polyneoptera.
We estimated rates based on time calibrated phylogenies and biologically realistic models of chromosome evolution in four orders (Mantodea, Blattodea (including Isoptera), Phasmatodea and Othoptera) of insects.
We further looked at the evolution of chromosome number by applying the chromPlus model that allows for a binary trait to be associated with differential rates of chromosome evolution with the sexual system database for the insects of order Phasmatodea.
Our results suggest fundamentally different rates of chromosome evolution in some clades and unexpected impacts of transitions to asexual reproduction in Phasmatodea. 
We also find evidence for an important for both fusions and fission in the origin of sex limited chromosomes and complex sex chromosome systems.