For this reason we hypothesize that asexual species will have higher rates of chromosome number evolution. 

Therefor we predict that asexual species should show increased rates of chromosome number evolution when compared to sexual species.


Chromosome number in this order is as low as 8 in the genera \textit{Hemimerus} and \textit{Dichroplus} and as high as 98 in the species \textit{Mastotermes darwiniensis}.
Of the 359 genera in our dataset we find that 64 are variable for chromosome number with the genera \textit{Sipyloidea} in the order Phasmatodea being particularly variable having a range of 22 to 80.


Furthermore, Polyneoptera includes XX/XO and XX/XY sex determination systems \citep{blackmon2016}. 
There is no a proper documentation on the impact of sex system on chromosome number changes in this group despite abundant data and variation. 

To explore dynamics of chromosome evolution we analyzed 783 karyotypes from the insect clade Polyneoptera.
We estimated rates based on time calibrated phylogenies and biologically realistic models of chromosome evolution in four orders (Mantodea, Blattodea (including Isoptera), Phasmatodea and Othoptera) of insects.
We further looked at the evolution of chromosome number by applying the chromPlus model that allows for a binary trait to be associated with differential rates of chromosome evolution with the sexual system database for the insects of order Phasmatodea.
Our results suggest fundamentally different rates of chromosome evolution in some clades and unexpected impacts of transitions to asexual reproduction in Phasmatodea. 
We also find evidence for an important for both fusions and fission in the origin of sex limited chromosomes and complex sex chromosome systems.



The insect order Blattodea showed an odd distribution suggesting that there is lot of variation for chromosome number evolution within the clade.
When we look at this clade we see that it includes both solitary roaches and eusocial termites. 
We investigated whether there were differences in chromosome number evolution rates associated with this stark difference in natural history and we find that eusocial termites (Isoptera) had significantly low rates of both fusion and fission than solitary roaches (Blattodea sensu stricto).
However, we do not find a significant difference in the rates of polyploidy within these to clades  .

In the insect orders, Mantodea, Orthoptera and Phasmatodea we find that Orthoptera showed lower rates for chromosome fusion and fission, moderate rates were seen in Mantodea and high rates were observed in Phasmatodea.
However, we do find in Mantodea, broader and higher rates were inferred for polyploidy.
We think that in Mantodea, despite having over 40 taxa, the data is not informative to generate acurate estimates for chromosome evolution rates.

Overall, We found that the Orders Orthoptera and Phasmatodea have higher polyploidy rates (95\% HPD: 0.0034 - 0.175 and 0.006 - 0.074 respectively) compared  to Blattodea (95\% HPD: 0 - 0.005).
Furthermore, Phasmatodea had a higher chromosome fusion rate (95\% HPD: 0.275 - 0.639) than Blattodea (95\% HPD: 0.04 - 0.243) and Orthoptera (95\% HPD: 0.001 - 0.041).
Blattodea on the other hand, had a higher fission rate (95\% HPD: 0.063 - 0.257) compared to Orthoptera (95\% HPD: 0 - 0.008)(\cref{tab:HPD} and \cref{fig:rates}).
We did not find significant differences in the chromosome evolution rates in the order Mantodea compared with other orders because this order had broader rates for chromosome evolution (95\% HPD fission: 0-0.11998; 95\% HPD fusion: 0 - 0.27214; 95\% HPD polyploidy 0 - 0.25006).





