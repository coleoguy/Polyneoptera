\section{Discussion}
The evolution of chromosome number across large clades and long time spans is fundamental to the diversity of genomes we observe across the tree of life.
Despite this we are only beginning to understand how chromosome number evolves.
In this study we have focused on the dynamics of chromosome evolution in Polyneoptera. 
Our analyses have revealed that this clade originated from an XO ancestor and that fusions have led to many new XY systems, and that fissions have led to many multiple sex chromosome systems.
In contrast to past findings \citep{ross2015} we see a rapid decline in the rate of chromosome evolution that coincides with the origin of eusocial Blattodea (Isoptera).
In Phasmatodea we find that transitions in to parthenogenetic reproduction allow for high rate of polyploidy but no significant increase in rates of dysploidy.

Transition between sex chromosome systems from X0 to XY can occur through X chromosome autosome fusions or by sex chromosome turnover with fixation of the ancestral X as an a new autosome.
Transitions via fusion will lead to a reduction in the total number of chromosomes while turnover should lead to an increase in the total number of chromosomes.
The data that we have collected shows a clear pattern of reduced chromosome number in taxa with recent XO to XY transitions (\cref{tab:fusions}).
Transition between sex chromosome systems from XY to multi-XY can occur through sex chromosome autosome fusions or by sex chromosome fission.
Transitions via fusion will lead to a reduction in the total number of chromosomes while fissions should lead to an increase in the total number of chromosomes.
In contrast to the pattern observed in XO to XY transitions, transitions from XY to multi-XY show a clear pattern of increased chromosome number in taxa with recent XY to multi-XY transitions (\cref{tab:fusions}).
These sex chromosome system transitions could be neutral and fix simply due to drift or sexually antagonistic variation on autosome may lead to positive selection for increasing portions of the genome becoming sex-linked \citep{charlesworth1980, kitano2012}.
So while our results from XO to XY transitions are consistent with an important role for fusions and potentially common sexual antagonism our results for fissions are not and thus we don't interpret changes in chromosome number through fusions being particularly driven by sexual antagonism. 

Our assembled data can also help us understand the fate of Y chromosomes. 
It has been suggested that Y chromosomes may be destined to decay and loss given enough time given the inevitability of mutation accumulation due to lack of recombination \citep{steinemann2005}.
Alternatively they may be retained through cycles of rejuvenation or even transitions into alternative forms of meiosis \citep{blackmon2015bioessay}.
In our inference of sex chromosome system evolution we performed stochastic mappings that allow us to estimate the rate of Y chromosome loss and the number of inferred losses in our dataset.
The rate of Y chromosome gains and Y chromosome losses are are both approximately 0.002 (\cref{tab:simmap.summary}). % add ref supptable here
However, we find that Y chromosome gains are more common with a mean of 15.3 while losses are relatively rare with mean of 6.7.
This pattern is intuitive when we consider that the ancestor of this group was likely XO and thus their has been relatively little time for the gain of Y chromosome to then be followed by its decay and loss.% make a supp table that has 95% credible interval for rates of XO to XY and XY to XO and # of transitions XO to XY and XY to XO

Karyotype evolution, particularly at the autosomal level can occur due to various forces.
It has been shown that female meiotic drive, where females preferentially segregate metacentric or telocentric chromosomes, is an important factor in the karyotype evolution in mammals (\citep{de2001female, blackmon2019meiotic}).
In addition, epistatic interactions between genes in two different chromosomes, can also lead to fusions between autosomes. 
On the other hand, the effective population size is also important in karyotype evolution.
When the effective size of a population is low the chromosome number can be altered by drift. 
Such changes would be highly deleterious or could be neutral. 
In contrast, chromosome number change due to natural selection should occur in large populations and only when such changes are beneficial.
In a study which looked at the speciation and karyotype evolution in mammals \citep{bush1977rapid}, the authors show that in species with low effective population size, the karyotype diversity is much higher.
They also show that, in species with higher effective population size, the karyotype diversity was low.

Karyotype evolution can also be influenced by life history traits. 
In our analysis of order level chromosome evolution, we find that Isoptera, which was thought as a separate clade but now is classified as highly derived clade within Blattodea , shows low rates of chromosome evolution suggesting that transition into eusocial lifestyle shows a marked reduction in chromosome number evolution.
However, the opposite pattern is seen in hymenoptera were eusocial insects show a marked increase in the chromosome evolution rate \citep{ross2015}.

Although, most of our trees supported for an importance in polyploidy, we don'At find higher rates of polyploidy within these orders.
In fact, it has been shown that polyploidy is much less frequent in animals that it is in plants (cite).
However, we do find two orders, Phasmatodea and Orthoptera, to have higher rates of polyploidy.
Phasmatodea is a special clade because this is the only clade of the studied clades where we find parthenogenetic species. 
In our analysis of how reproductive mode influence chromosome number evolution we found that there is no significant difference in the rates of fusion and fission between the two reproductive modes, sexual and parthenogenetic. 
However, we did find that the rate of polyploidy is much higher in parthenogenetic species than in sexually reproducing species. 
This result is in line with what we see in insects where, polyploidy is commonly associated with parthenogenetic reproduction \citep{lokki1980polyploidy}.
We think that it is because of dosage balance we see that there is no significant difference in the chromosome fusion and fission rates in both parthenogenetic and sexually reproducing species. 

Overall our results suggests that chromosome evolution in the clade polyneoptera is highly idiosyncratic and likely impacted by variety of forces.