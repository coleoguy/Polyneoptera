\section{Discussion}

The evolution of chromosome number across large clades and long time spans is fundamental to the diversity of genomes we observe across the tree of life.
Despite this we are only beginning to understand how chromosome number evolves.
In this study we have focused on the dynamics of chromosome evolution in Polyneoptera. 
Our analyses have revealed that this clade originated from an XO ancestor and that fusions have led to many new XY systems, and that fissions have led to many multiple SCSs.
We see a rapid decline in the rate of chromosome evolution that coincides with the origin of eusocial Blattodea (Isoptera).
This is the opposite of the pattern documented in Hymenopera \citep{ross2015}.
In Phasmatodea we find that transitions in to parthenogenetic reproduction allow for high rate of polyploidy but no significant increase in rates of dysploidy.

\subsubsection{Sex chromosomes and chromosomes number evolution}
Transition between SCSs from XO to XY can occur through X chromosome autosome fusions or by sex chromosome turnover with fixation of the ancestral X as an a new autosome.
Transitions via fusion will lead to a reduction in the total number of chromosomes while turnover should lead to an increase in the total number of chromosomes.
The data that we have collected shows a clear pattern of reduced chromosome number in taxa with recent XO to XY transitions (\cref{tab:fusions}).
Transition between SCSs from XY to multi-XY can occur through sex chromosome autosome fusions or by sex chromosome fission.
Transitions via fusion will lead to a reduction in the total number of chromosomes while fissions should lead to an increase in the total number of chromosomes.
In contrast to the pattern observed in XO to XY transitions, transitions from XY to multi-XY show a clear pattern of increased chromosome number in taxa with recent XY to multi-XY transitions (\cref{tab:fusions}).
These SCS transitions could be neutral and fix simply due to drift or sexually antagonistic variation on autosome may lead to positive selection for increasing portions of the genome becoming sex-linked \citep{charlesworth1980, kitano2012}.
So while our results from XO to XY transitions are consistent with an important role for fusions and potentially common sexual antagonism our results for fissions are not and thus we don't interpret changes in chromosome number through fusions being solely driven by sexual antagonism. 

Our assembled data can also help us understand the fate of Y chromosomes. 
It has been suggested that Y chromosomes may be destined to decay and loss given enough time given the inevitability of mutation accumulation due to lack of recombination \citep{steinemann2005}.
Alternatively they may be retained through cycles of rejuvenation or even transitions into alternative forms of meiosis \citep{blackmon2015bioessay}.
In our inference of SCS evolution we performed stochastic mappings that allow us to estimate the rate of Y chromosome loss and the number of inferred losses in our dataset.
The rate of Y chromosome gains and Y chromosome losses are are both approximately 0.002 (\cref{tab:simmap.summary}). 
However, we find that Y chromosome gains are more common with a mean of 15.3 while losses are relatively rare with mean of 6.7.
This pattern is intuitive when we consider that the ancestor of this group was likely XO and thus their has been relatively little time for the gain of Y chromosome to then be followed by its decay and loss.

\subsubsection{Eusociality and chromosome number}
In our analysis of Blattodea, we inferred a significantly lower rate of chromosome evolution in the subclade Isoptera than the rest of Blattodea (roaches).
In the past it has been hypothesized that large rate differences for instance those seen among some mammal clades might be explained by differences in effective population size \citep{bush1977rapid}.
Future work that examined effective population sizes of solitary Blattodea and eusocial Isoptera might reveal whether the observed rate differences in these lineages is consistent with this earlier hypothesis that large effective population size is associated with lower rates of chromosome evolution.
It should be noted though that more recent phylogenetically informed analysis of over 1000 species of mammals showed that female meiotic drive may be a primary driver of chromosome number evolution in mammals \citep{blackmon2019meiotic}.
Furthermore, the rate difference that we infer among the eusocial and solitary lineages of Blattodea are in the opposite direction of those observed between eusocial and solitary Hymenoptera \citep{ross2015}.
This disconnect even among insect clades points to the need for more taxonimcally diverse analyses that can disentangle the roll that life history and effective population size may have on the evolution of chromosome number.

\subsubsection{Constraints on chromosome number evolution}
Although we have discussed increases and decreases in chromosome number multiple mechanisms may underlie these changes.
In many clades chromosome number is likely to change primarily through fusions and fissions of existing chromosomes \citep{sved2016, blackmon2019meiotic}.
But, chromosome number could also increase due to aneuploidy events that fix in a population creating duplicate copies of one or more chromosomes.
The converse chromosome number decrease due to aneuploidy is likely to be exceedingly rare since all genes on the chromosome would have to be dispensable.
When an entire chromosome is duplicated through aneuploidy all genes that are on the chromosome are then present in multiple copies and the expectation is to rapidly loose the vast majority of these duplicated genes \citep{ohno}.
Whether the copy on the duplicated or the original chromosome is lost should be largely random.
This should lead to a subset of genes being retained on both the ancestral and duplicated chromosome.
However, this process could lead to sterile offspring if two populations (one with the chromosome duplication and one without) hybridize.
Since the heterozygous offspring may have difficulty segregating unmatched chromosomes during meiosis \citep{white1978}.
However, asexually reproducing species should be immune from this potential problem since they cannot outcross to an individual with a different chromosome compliment. 
In both sexual and asexual species chromosome increase due to aneuploidy may be rare due to the impact of aneuploidy on dosage which may lead to stoichiometric imbalances in gene networks with interacting products on different chromosomes.
For these reasons we expected to see a higher rate of chromosome number increase and decrease in asexual species since difficulties in mating with individuals with different chromosome compliments would be relaxed.
Our Phasmatodea dataset has a mean of 9.3 transitions from sexual to asexual reproduction and offers a chance to test this hypothesis (\cref{tab:phas.simmap.summary}).
To our surprise our analysis illustrates that rates of chromosome increase and decrease are equal in sexual and asexual Phasmatodea (\cref{fig:phas.plot}).
We interpret this as evidence that the constraints on chromosome number change via fusions, fissions, and aneuploidy are largely similar in these two groups. 
The most parsimonious explanation seems to be that these changes are largely neutral and that individuals that are heterozygous for chromosomal rearrangements do not typically have difficulty properly segregating chromosomes during meiosis.

\subsubsection{Variation in rates of chromosome number evolution}
Most studies of chromosome evolution have been done in isolation on small clades \citep{rockman2002, mccann2016, deoliveira}. 
This creates a challenge in understanding variation in rates of chromosome number evolution across the tree of life since rates are fundamentally influenced by the time constraints and branch lengths inferred in a study (but see: \citealt{blackmon2019meiotic, zenil2017}).
By inferring rates in four orders all using a common tree we are able to make a more valid comparison among clades and determine whether some groups are evolving at significantly different rates.
We found many examples of significantly different rates of chromosome evolution among orders.
Blattodea (including or excluding Isoptera) have a higher rate of increases than Orthoptera (\cref{tab:HPD}).
Blattodea (without Isoptera) have a higher rate of decrease than Isoptera and Orthoptera  (\cref{tab:HPD}).
Decreases are also higher in Phasmatodea than Blattodea (including Isoptera), Isoptera, and Orthoptera  (\cref{tab:HPD}).
Polyploidy is higher in Orthoptera than Blattodea (including Isoptera) and, Phasmatodea is higher than Blattodea (including Isoptera)  (\cref{tab:HPD}).
In line with existing evidence \citep{lokki1980polyploidy, blackmon2016} polyploidy is higher in asexual than sexual species (\cref{fig:phas.plot}).
In most of these cases it is unclear what leads to these significant differences in rates of chromosome number evolution.

One possible explanation for variation in rates of chromosome evolution is fundamental differences in the repeat content of the genome.
For instance, large numbers or recent expansions of transposable elements may lead to more frequent chromosome breakage or other structural rearrangements that change chromosome number \citep{carbone2014gibbon}.
If transposable elements have expanded in the genomes of a clade we might expect to see a signature of this in increased genome sizes \citep{kidwell2002transposable}.
However, we found no significant association between genome size and rates of chromosome evolution suggesting that repetitive content is not a driving force in the stability of large scale genome structure across Polyneoptera.
Moving forward, the recent development of multiple probabilistic models of chromosome evolution that allow for associations with speciation or binary characters offers a way forward to further tease apart the determinants of rates of chromosome evolution \citep{freyman2018, zenil2018chromploid, blackmon2019meiotic}

Taken as a whole our results illustrate that changes in the rate of chromosome evolution are widespread across large taxonomic groups.
This has important implications for our understanding of speciation processes.
For instance while many chromosomal speciation models \citep{baker1986, white} have been thought to be unimportant in recent times it may be that in groups like Blattodea with high rates of chromosomal evolution these models may be an important source of extant diversity.
More broadly depending on the importance of epistatic relationships the reorganization of the genome through fusions and fissions may be important in determining the ability of organisms to adapt to novel environments.
