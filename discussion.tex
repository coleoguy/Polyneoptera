\section{Discussion}
Our goal on this study was to understand the dynamics of chromosome evolution in the insect clade Polyneoptera. 
First we show that the ancestor for the Polyneoptera clade had X0 sex chromosome system. 
Then we show that fusions play a key role in transitions from XO to XY while fission play a key role in transitions from XY to complex sex chromosome system.
Our results on chromosome number evolution, suggests that there are striking differences in the tempo and mode of chromosome evolution within this clade.
We showed that in some orders within Polyneoptera, polyploidy was more dominant, and, in some orders, it was fusions and fissions that controlled the mode of chromosome evolution.
We also find that transition from solitary to eusocial lifestyle in the order Blattodea, marks a striking reduction in the fusion and fission rates.
Furthermore, we showed that parthenogenesis is associated with increased rates of polyploidy but not disploidy.
Finally, in our ancestral states reconstructions, we showed that in the order Orthoptera, which had a higher polyploidy rate, there was a weak support for ancestral polyploidy event. 

Transition between sex chromosome systems (e.g. from X0 to XY or ZO to ZW [here we are focusing on XY system as the clade Polyneoptera do not have ZW sex chromosome system.]) can occur through fusions and/or fissions.
A fusion event between the X chromosome in an X0 sex chromosome system and an autosome will result in a sex chromosome transition from X0 sex chromosome system to XY sex chromosome system. In such an event, the unfused autosome will become the neo-Y chromosome. 
Complex sex chromosome systems can arise from an ancestral XY sex chromosome system through a fusion or a fission event. 
For example, A fusion between an autosome and an X chromosome in an XY sex chromosome system will result in and XYY sex chromosome system. 
On the other hand, a fission event in the X chromosome in an XY sex chromosome system will result in XXY sex chromosome system. 
Recent events of fusion of sex chromosomes with autosomse to generate neo-sex chromosomes has been reported in glass knife fishes \citep{henning2011independent} and in Japanese sea sticklebacks \citep{kitano2012}.

Transitions from XY sex chromosome system in to X0 sex chromosome system is rare in our phylogenetic tree. 
We find only three instances in our phylogeny where transitions back to XO sex chromosome system from XY sex chromosome system had occurred. 
This could occur in two ways.
Given enough time the Y chromosome will degrade due to mutation accumulation due to lack of recombination and eventually will be lost.
The other way that this could happen is by fusion or translocation of the Y chromosome with the X chromosome or even with an autosome. 

Karyotype evolution, particularly at the autosomal level can occur due to various forces.
It has been shown that female meiotic drive, where females preferentially segregate metacentric or telocentric chromosomes, is an important factor in the karyotype evolution in mammals (\citep{de2001female, blackmon2019meiotic}).
In addition, epistatic interactions between genes in two different chromosomes, can also lead to fusions between autosomes. 
On the other hand, the effective population size is also important in karyotype evolution.
When the effective size of a population is low the chromosome number can be altered by drift. 
Such changes would be highly deleterious or could be neutral. 
In contrast, chromosome number change due to natural selection should occur in large populations and only when such changes are beneficial.
In a study which looked at the speciation and karyotype evolution in mammals \citep{bush1977rapid}, the authors show that in species with low effective population size, the karyotype diversity is much higher.
They also show that, in species with higher effective population size, the karyotype diversity was low.

Karyotype evolution can also be influenced by life history traits. 
In our analysis of order level chromosome evolution, we find that Isoptera, which was thought as a separate clade but now is classified as highly derived clade within Blattodea , shows low rates of chromosome evolution suggesting that transition into eusocial lifestyle shows a marked reduction in chromosome number evolution.
However, the opposite pattern is seen in hymenoptera were eusocial insects show a marked increase in the chromosome evolution rate \citep{ross2015}.

Although, most of our trees supported for an importance in polyploidy, we don'At find higher rates of polyploidy within these orders.
In fact, it has been shown that polyploidy is much less frequent in animals that it is in plants (cite).
However, we do find two orders, Phasmatodea and Orthoptera, to have higher rates of polyploidy.
Phasmatodea is a special clade because this is the only clade of the studied clades where we find parthenogenetic species. 
In our analysis of how reproductive mode influence chromosome number evolution we found that there is no significant difference in the rates of fusion and fission between the two reproductive modes, sexual and parthenogenetic. 
However, we did find that the rate of polyploidy is much higher in parthenogenetic species than in sexually reproducing species. 
This result is in line with what we see in insects where, polyploidy is commonly associated with parthenogenetic reproduction \citep{lokki1980polyploidy}.
We think that it is because of dosage balance we see that there is no significant difference in the chromosome fusion and fission rates in both parthenogenetic and sexually reproducing species. 

Overall our results suggests that chromosome evolution in the clade polyneoptera is highly idiosyncratic and likely impacted by variety of forces.