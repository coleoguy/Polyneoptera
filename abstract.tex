\section{Abstract}
To help address this issue, we have assembled a dataset of 783 karyotypes from species in the insect group Polyneoptera. 
We then built time-calibrated phylogenies and applied biologically realistic probabilistic models of chromosome evolution.
Our analysis reveals that fusions are responsible for the transition from XO to XY sex chromosome systems and that fissions play an essential role in the origin of multiple sex chromosomes systems (e.g., XXY).
We also find that many closely related clades exhibit striking differences in rates of chromosome evolution.
In particular, termites have significantly lower rates of chromosome number evolution in comparison to the rest of Blattodea.
We also find that asexuality is associated with increased rates of whole genome duplication but not fusions, fissions, or aneuploidy.







