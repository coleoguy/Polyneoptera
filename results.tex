\section{Results}

\subsection{Evolution of sex chromosome systems}
\subsubsection{Processes responsible for transitions in sex chromosome systems}
In our data set 21 genera contain species with different types of sex chromosome systems (i.e. XO XY, and or multi-XY).
In each of these genera we calculate the mean number of chromosome for all species with a given type of sex chromosome.
By comparing these means within genera we can determine if differences are consistent with fusions or fissions as a source of transitions among sex chromosome systems.
Briefly, if transitions from XO to XY are generated by the fusion of an autosome to a sex chromosome we would expect a lower mean chromosome number for XY species.
Likewise, if transitions from XY to multi-XY are generated by the fusion of an autosome to a sex chromosome we would expect a lower mean chromosome number for multi-XY species.
In contrast, if transitions from XY to multi-XY are generated by the fission of a sex chromosome we would expect a higher mean chromosome number for multi-XY species.

We find strong support for fusions as a source of transitions from XO to XY sex chromosome systems.
Of the 15 genera with both XO and XY species 94\% (15/16) show a lower mean chromosome number in XY species (\cref{tab:fusions}). 
However, we find little support for fusions leading to transitions from XY to multi-XY.
In fact, only one of the seven genera with both XY and multi-XY has a lower mean chromosome number for multi-XY species.
Instead, 71\% (5/7) of the genera with both XY and multi-XY have a higher mean chromosome number in multi-XY species.
This analysis is limited to only those genera with varaition in sex chromosome systems and thus omits much our our data.
Examining the distribution of chromosome number among all species in each order parsed by sex chromosome system suggests that in some groups the origin of transitions may differ.
For instance, the mean of all XY species in both Blattodea and Dermaptera is higher than the mean of XO species in both orders.
This pattern is not expected if fussions are the primary source of transitions from XO to XY \cref{fig:order.plots}.

\begin{table}
\begin{tabular}{llccccc}
\hline
\multirow{2}{*}{Order} & \multirow{2}{*}{Genus} & \multicolumn{3}{l}{\begin{tabular}[c]{@{}c@{}}Mean number of\\ chromosomes\end{tabular}} & \multicolumn{2}{l}{Evidence supports} \\ \cline{3-7} 
                       &                        & XO                          & XY                          & multi                        & Fusion            & Fission           \\ \hline
Orthoptera             & Aleuas                 & 10                          & 10.2                        & -                            & -                 & -                 \\
Orthoptera             & Dichroplus             & 11.7                        & 9.7                         & 11                           & +                 & +                 \\
Orthoptera             & Diponthus              & 11.5                        & 11                          & -                            & +                 & -                 \\
Orthoptera             & Eurotettix             & -                           & 11                          & 11                           & -                 & -                 \\
Orthoptera             & Gryllotalpa            & 10.7                        & 6                           & -                            & +                 & -                 \\
Orthoptera             & Leiotettix             & 12                          & 9                           & 7.5                          & +                 & -                 \\
Orthoptera             & Scotussa               & 11.6                        & 9.5                         & 11                           & +                 & +                 \\
Orthoptera             & Scyllina               & 12                          & 11                          & -                            & +                 & -                 \\
Orthoptera             & Tetrixocephalus        & 12                          & 11                          & -                            & +                 & -                 \\
Orthoptera             & Xyleus                 & 12                          & 11                          & -                            & +                 & -                 \\
Orthoptera             & Zoniopoda              & 12                          & 11                          & -                            & +                 & -                 \\
Phasmatodea            & Didymuria              & 18.6                        & 15.4                        & -                            & +                 & -                 \\
Phasmatodea            & Isagoras               & 19                          & 17                          & -                            & +                 & -                 \\
Phasmatodea            & Leptynia               & 19.2                        & 18                          & -                            & +                 & -                 \\
Phasmatodea            & Podacanthus            & 18                          & 14                          & -                            & +                 & -                 \\
Phasmatodea            & Prisopus               & 25                          & 14                          & -                            & +                 & -                 \\
Mantodea               & Deiphobe               & 10                          & -                           & 14                           & -                 & -                 \\
Dermaptera             & Forficula              & -                           & 12                          & 12.8                         & -                 & +                 \\
Dermaptera             & Nala                   & -                           & 18.5                        & 19                           & -                 & +                 \\
Dermaptera             & Nesogaster             & 11                          & -                           & 11                           & -                 & -                 \\
Plecoptera             & Perla                  & 10.5                        & 5                           & 12.8                         & +                 & +                 \\ \hline
\end{tabular}
\caption{Chromsome number and sex chromosome systems. In each genus we report the mean number of chromosomes for all species having each type of sex chromosome systems. The fusion and fission columns contain a + to indicate a difference of chromosome number that is consistent with either fusions or fissions. A - symbol indicates a distribution of chromosome number that is uninformative.}
\label{tab:fusions}
\end{table}


\subsubsection{Ancestral states and rates of sex chromosome evolution}

We find that the ancestral state for sex chromosome system in polyneoptera clade was XO with a probability of 90.3\%.
Similarly, the most probable ancestral state for of each order was also XO, with the exception of Isoptera and Dermaptera where XY is more probable.
(\cref{fig:sex.asr.plot}).
We find that the transition rates from XO to XY and ..... % HB say something about rates 
Analysis of stochastic mappings show that.... %HB do stochastic mapping 

\subsection{Karyotype variation}
We find a significant difference in variance in chromosome numbers among orders of Polyneoptera (Levene's test \textit{p}-value : 2.2e-16). %HB do posthocs
% On the other hand, Othoptera, which has the vast majority of the records in our dataset, had very little variation in chromosome number. In fact, 65\% of the order Orthoptera had 12 as the haploid chromosome number and 20\% had 11 as the haploid chromosome count. % probably add back into explain the posthoc results.
Much of these differences in variance are obvious when looking at the much smaller phylogenetic dataset in \cref{fig:phyloplot}.

\subsection{Rate inference}
Though these differences in variance suggest some orders are evolving more quickly we must control for the phylogeny to draw any rigorous conclusions.

We began with a base model that allows for fusions and fissions and compared this via a likelihood ratio test with a model that included polyploidy as well.
We found that XX\% of our likelihood ratio tests supported an important role for polyploidy.
For this reason all analyses were done with a model that had fusions, fissions, and polyploidy.

% for each of these write sentence report mean of each parameter
In Blattodea we estimate a mean fusion rate of XXX.XX, a fission rate of XXX.XX, and a polyploidy rate of XX.XXX (\cref{tab:HPD}).

In contrast if we remove isoptera from blattodea then we find higher/lower rates of .....

Isoptera

Mantodea

Phasmatodea

The lowest rates we estimated were in Othoptera where the fusion rate was XXX,X....


% other option talk about all fission, all fusions, all polyploidy...
% you decide

(\cref{fig:rates}). 

\begin{table}[ht]
\centering
\begin{tabular}{lccc}
\hline
\textbf{Order}          & \textbf{Fission} & \textbf{Fusion} & \textbf{Polyploidy} \\ \hline
Blattodea               & 0.063 - 0.257    & 0.04 - 0.243    & 0 - 0.005           \\
Blattodea sensu stricto & 0.173 - 0.604    & 0.202 - 0.619   & 0 - 0.01            \\
Isoptera                & 0.002 - 0.1      & 0.019 - 0.114   & 0 - 0.006           \\
Mantodea                & 0 - 0.329        & 0 - 0.737       & 0 - 0.69            \\
Orthoptera              & 0 - 0.008        & 0.01 - 0.041    & 0.034 - 0.175       \\
Phasmatodea             & 0 - 0.342        & 0.275 - 0.639   & 0.006 - 0.074       \\ \hline
\end{tabular}
\caption{95\% Highest posterior density distribution for chromosome fissions, fusions and polyploidy of the four orders}
\label{tab:HPD}
\end{table}%HB same number of digits

\subsection{Ancestral state reconstruction}
Our analysis of Orthoptera suggests six is the most probable chromosome number for the most recent common ancestor of taxa in our analysis. 
However, a chromosome number of three is also given a high probability on some ancestral state reconstructions suggesting that at least some phylogenies support an early whole genome duplication in this clade.
However, this ancestral whole genome duplication was supported only in 19\% of the sampled trees.
For all the other orders, we observed a wider range for the most probable chromosomes number for the most recent common ancestor.
In the order Mantodea, the most probable chromosome number ranged from six to nine with 31\% of the sampled trees supporting six, 10\% supporting seven, 36\% supporting eight and 23\% supporting nine as the ancestral state of the root in this order. 
For Blattodea, the most probable chromosome number ranged from 12 to 17 with 23\% of the sampled trees supporting 12, 30\% supporting 13, 26\% supporting 14, 16\% supporting 15, 3\% supporting 16 and 2\% supporting 17 as the ancestral state of the root in this order.
For Isoptera, the most probable chromosome number ranged from 19 to 23.
For Phasmatodea, 98\% of the sampled trees supported nine as the most probable chromosome number at the root and 2\% of the trees supported 18 and 19 as the root state. 
However, in Phasmatodea, these ancestral states had much lower probability (\cref{fig:asr}).

\subsection{Mode of reproduction and chromosome rates}
In our dataset, we find 13 species which are parthenogenesis. 
All these species were seen in the clade Phasmatodea.
Therefore we tested whether the mode of reproduction is associated with chromosome number evolution. 
For this purpose we redoubled our efforts to increase the number of species for this clade and inferred a second phylogeny (figure of Phasmatodea phylogeny).
This increased our samples from 28, which we had in Polyneoptera data set, to 41 species. 
In this analysis, we found that there is no significant difference in the rates chromosome fusion and fission between sexually and asexually reproducing species.
However, we find that rates of polyploidy is significantly high in asexually reproducing species than in sexually reproducing species (\cref{fig:phas.plot}).