\section{Results}

\subsection{Variation in sex chromosome systems}
We classified the sex chromosome systems to three classes, X0, XY and multi-XY.
If the sex chromosome system of a particular species included multiple sex chromosomes (e.g: XYY or XXXY) we classified these species as having multi-XY sex chromosome system.
We reconstructed the ancestral states for the sex chromosome system of the polyneoptera clade to infer patterns of sex chromosome system evolution.
We find that the ancestral state for sex chromosome system in polyneoptera clade was X0 with a probability of 90.3\%.
When we look at the order level ancestral states, we find that except for orders Isoptera and Dermaptera, ancestral state of the sex chromosome system for all other clades was X0 as well (\cref{fig:sex.asr.plot}).
We also find that the transition from XY sex chromosome system to X0 sex chromosome system is rare in the Polyneoptera clade. 
There were only three such instances were we see the transition from XY sex chromosome system to XO sex chromosome system.
Additionally, we compared the number of chromosomes among species in genera with more than one sex chromosome system to see whether the differences in chromosome number suggests that fusion or fission have been important in generating the variation in sex chromosome systems that we find (\cref{fig:order.plots} and \cref{tab:fusions}).
Here we have looked at the mean chromosome number for each sex chromosome system in these genera.
We find that 94\% of genera with both X0 and XY sex chromosome systems having a higher mean chromosome number in X0 sex chromosome system than in XY sex chromosome system suggesting that an important role for fusions in transition from X0 sex chromosome system to XY sex chromosome system.
We only find a single genus which shows support for a role of fusions in transition from XY sex chromosome system to multi-XY sex chromosome system.
On the other hand, 71\% of the genera with both XY and multi-XY systems, we find a higher mean chromosome number in multi-XY sex chromosome system than in XY sex chromosome system showing support for a role of fission in transition from XY sex chromosome system to multi-XY sex chromosome system.

\begin{table}
\begin{tabular}{llccccc}
\hline
\multirow{2}{*}{Order} & \multirow{2}{*}{Genus} & \multicolumn{3}{l}{\begin{tabular}[c]{@{}c@{}}Mean number of\\ chromosomes\end{tabular}} & \multicolumn{2}{l}{Evidence supports} \\ \cline{3-7} 
                       &                        & XO                          & XY                          & multi                        & Fusion            & Fission           \\ \hline
Orthoptera             & Aleuas                 & 10                          & 10.2                        & -                            & -                 & -                 \\
Orthoptera             & Dichroplus             & 11.7                        & 9.7                         & 11                           & +                 & +                 \\
Orthoptera             & Diponthus              & 11.5                        & 11                          & -                            & +                 & -                 \\
Orthoptera             & Eurotettix             & -                           & 11                          & 11                           & -                 & -                 \\
Orthoptera             & Gryllotalpa            & 10.7                        & 6                           & -                            & +                 & -                 \\
Orthoptera             & Leiotettix             & 12                          & 9                           & 7.5                          & +                 & -                 \\
Orthoptera             & Scotussa               & 11.6                        & 9.5                         & 11                           & +                 & +                 \\
Orthoptera             & Scyllina               & 12                          & 11                          & -                            & +                 & -                 \\
Orthoptera             & Tetrixocephalus        & 12                          & 11                          & -                            & +                 & -                 \\
Orthoptera             & Xyleus                 & 12                          & 11                          & -                            & +                 & -                 \\
Orthoptera             & Zoniopoda              & 12                          & 11                          & -                            & +                 & -                 \\
Phasmatodea            & Didymuria              & 18.6                        & 15.4                        & -                            & +                 & -                 \\
Phasmatodea            & Isagoras               & 19                          & 17                          & -                            & +                 & -                 \\
Phasmatodea            & Leptynia               & 19.2                        & 18                          & -                            & +                 & -                 \\
Phasmatodea            & Podacanthus            & 18                          & 14                          & -                            & +                 & -                 \\
Phasmatodea            & Prisopus               & 25                          & 14                          & -                            & +                 & -                 \\
Mantodea               & Deiphobe               & 10                          & -                           & 14                           & -                 & -                 \\
Dermaptera             & Forficula              & -                           & 12                          & 12.8                         & -                 & +                 \\
Dermaptera             & Nala                   & -                           & 18.5                        & 19                           & -                 & +                 \\
Dermaptera             & Nesogaster             & 11                          & -                           & 11                           & -                 & -                 \\
Plecoptera             & Perla                  & 10.5                        & 5                           & 12.8                         & +                 & +                 \\ \hline
\end{tabular}
\caption{Chromsome number and sex chromosome systems. In each genus we report the mean number of chromosomes for all species having each type of sex chromosome systems. The fusion and fission columns contain a + to indicate a difference of chromosome number that is consistent with either fusions or fissions. A - symbol indicates a distribution of chromosome number that is uninformative.}
\label{tab:fusions}
\end{table}


\subsection{Karyotype variation}
We find a significant variation in the chromosome numbers within the insect clade Polyneoptera (Levene's test \textit{p}-value : 2.2e-16).
In particular, orders Blattodea, Dermaptera and Phasmatodea had a much higher variance in chromosome number compared to the rest of the group.
In addition, Plecoptera and Mantodea also showed a considerable variation in chromosome number. 
On the other hand, Othoptera, which has the vast majority of the records in our dataset, had very little variation in chromosome number. 
In fact, 65\% of the order Orthoptera had 12 as the haploid chromosome number and 20\% had 11 as the haploid chromosome count. 
Together with Orthoptera, the insect order Embiidina had a considerably low variation in chromosome counts as well. 
We did not have enough data to get a general idea about the karyotype variation in the order Notoptera as it only had three records for karyotype data (\cref{fig:phyloplot}).

\subsection{Rate inference}
Based on the clear evidence that these orders are significantly different from each other in terms of the karyotipic variation, we did an order level analysis to understand the tempo and mode of chromosome evolution within these orders (\cref{fig:rates}). 
To get reliable estimates of the chromosome evolution, we only used orders with more than 20 taxonomic units.
This criterion reduced our analysis to four orders which included Blattodea, Mantodea, Orthoptera and Phasmatodea.

We used the ChromPlus model, to infer the chromosome number evolution rates (\citep{blackmon2019meiotic}).
The parametes for chromosome evolution in this model included, chromosome fusion, chromosome fission, demiploidy and polyploidy.
In our model we did not used demiploidy as a possible way of chromosome number evolution as there is little support for this type of changes in insects.
To see if polyploidy plays an important role in chromosome number evolution in these clades, we did a likelihood ratio test were we compared a simpler model which only included fusion and fission with a more complex model which included fusion, fission and polyploidy.
A non trivial number of trees supported for the more complex model and therefore we used the model with all three parameters to infer the chromosome number evolution rates in these clades (include the LRT plot).

The insect order Blattodea showed an odd distribution suggesting that there is lot of variation for chromosome number evolution within the clade.
When we look at this clade we see that it includes both solitary roaches and eusocial termites. 
We investigated whether there were differences in chromosome number evolution rates associated with this stark difference in natural history and we find that eusocial termites (Isoptera) had significantly low rates of both fusion and fission than solitary roaches (Blattodea sensu stricto).
However, we do not find a significant difference in the rates of polyploidy within these to clades (\cref{tab:HPD}) .

In the insect orders, Mantodea, Orthoptera and Phasmatodea we find that Orthoptera showed lower rates for chromosome fusion and fission, moderate rates were seen in Mantodea and high rates were observed in Phasmatodea.
However, we do find in Mantodea, broader and higher rates were inferred for polyploidy.
We think that in Mantodea, despite having over 40 taxa, the data is not informative to generate acurate estimates for chromosome evolution rates.

Overall, We found that the Orders Orthoptera and Phasmatodea have higher polyploidy rates (95\% HPD: 0.0034 - 0.175 and 0.006 - 0.074 respectively) compared  to Blattodea (95\% HPD: 0 - 0.005).
Furthermore, Phasmatodea had a higher chromosome fusion rate (95\% HPD: 0.275 - 0.639) than Blattodea (95\% HPD: 0.04 - 0.243) and Orthoptera (95\% HPD: 0.001 - 0.041).
Blattodea on the other hand, had a higher fission rate (95\% HPD: 0.063 - 0.257) compared to Orthoptera (95\% HPD: 0 - 0.008)(\cref{tab:HPD} and \cref{fig:rates}).
We did not find significant differences in the chromosome evolution rates in the order Mantodea compared with other orders because this order had broader rates for chromosome evolution (95\% HPD fission: 0-0.11998; 95\% HPD fusion: 0 - 0.27214; 95\% HPD polyploidy 0 - 0.25006).

\begin{table}[ht]
\centering
\begin{tabular}{lccc}
\hline
\textbf{Order}          & \textbf{Fission} & \textbf{Fusion} & \textbf{Polyploidy} \\ \hline
Blattodea               & 0.063 - 0.257    & 0.04 - 0.243    & 0 - 0.005           \\
Blattodea sensu stricto & 0.173 - 0.604    & 0.202 - 0.619   & 0 - 0.01            \\
Isoptera                & 0.002 - 0.1      & 0.019 - 0.114   & 0 - 0.006           \\
Mantodea                & 0 - 0.329        & 0 - 0.737       & 0 - 0.69            \\
Orthoptera              & 0 - 0.008        & 0.01 - 0.041    & 0.034 - 0.175       \\
Phasmatodea             & 0 - 0.342        & 0.275 - 0.639   & 0.006 - 0.074       \\ \hline
\end{tabular}
\caption{95\% Highest posterior density distribution for chromosome fissions, fusions and polyploidy of the four orders}
\label{tab:HPD}
\end{table}%HB same number of digits

\subsection{Ancestral state reconstruction}
Our analysis of Orthoptera suggests six is the most probable chromosome number for the most recent common ancestor of taxa in our analysis. 
However, a chromosome number of three is also given a high probability on some ancestral state reconstructions suggesting that at least some phylogenies support an early whole genome duplication in this clade.
However, this ancestral whole genome duplication was supported only in 19\% of the sampled trees.
For all the other orders, we observed a wider range for the most probable chromosomes number for the most recent common ancestor.
In the order Mantodea, the most probable chromosome number ranged from six to nine with 31\% of the sampled trees supporting six, 10\% supporting seven, 36\% supporting eight and 23\% supporting nine as the ancestral state of the root in this order. 
For Blattodea, the most probable chromosome number ranged from 12 to 17 with 23\% of the sampled trees supporting 12, 30\% supporting 13, 26\% supporting 14, 16\% supporting 15, 3\% supporting 16 and 2\% supporting 17 as the ancestral state of the root in this order.
For Isoptera, the most probable chromosome number ranged from 19 to 23.
For Phasmatodea, 98\% of the sampled trees supported nine as the most probable chromosome number at the root and 2\% of the trees supported 18 and 19 as the root state. 
However, in Phasmatodea, these ancestral states had much lower probability (\cref{fig:asr}).

\subsection{Mode of reproduction and chromosome rates}
In our dataset, we find 13 species which are parthenogenesis. 
All these species were seen in the clade Phasmatodea.
Therefore we tested whether the mode of reproduction is associated with chromosome number evolution. 
For this purpose we redoubled our efforts to increase the number of species for this clade and inferred a second phylogeny (figure of Phasmatodea phylogeny).
This increased our samples from 28, which we had in Polyneoptera data set, to 41 species. 
In this analysis, we found that there is no significant difference in the rates chromosome fusion and fission between sexually and asexually reproducing species.
However, we find that rates of polyploidy is significantly high in asexually reproducing species than in sexually reproducing species (\cref{fig:phas.plot}).